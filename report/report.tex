\documentclass[10pt,a4paper,onecolumn]{article} % titlepage

\usepackage[margin=2cm]{geometry}

\usepackage{helvet}
\renewcommand{\familydefault}{\sfdefault}

\usepackage{graphicx}
\usepackage{amsmath}
\usepackage{amssymb}
\usepackage{booktabs}


% It is strongly recommended to use hyperref, especially for the review version.
% hyperref with option pagebackref eases the reviewers' job.
% Please disable hyperref *only* if you encounter grave issues, e.g. with the
% file validation for the camera-ready version.
%
% If you comment hyperref and then uncomment it, you should delete
% ReviewTempalte.aux before re-running LaTeX.
% (Or just hit 'q' on the first LaTeX run, let it finish, and you
%  should be clear).
\usepackage[pagebackref,breaklinks,colorlinks]{hyperref}


% Support for easy cross-referencing
\usepackage[capitalize]{cleveref}
\crefname{section}{Sec.}{Secs.}
\Crefname{section}{Section}{Sections}
\Crefname{table}{Table}{Tables}
\crefname{table}{Tab.}{Tabs.}


\begin{document}

\title{EEE3032 – Coursework Assignment \\ Visual Search of an Image Collection }

\author{Rohit Krishnan, {URN: 6839323},
{\tt r.00088@surrey.ac.uk}
}
\maketitle

\newpage

\begin{abstract}
Your abstract goes here...
...
\end{abstract}

{
  \hypersetup{linkcolor=black}
  \tableofcontents
}

\twocolumn

%%%%%%%%% BODY TEXT
\section{Introduction}
\label{sec:intro}

this book is awesome \cite{szeliski2011computer}

\newpage

%%%%%%%%% REFERENCES
{\small
\bibliographystyle{ieee_fullname}
\bibliography{egbib}
}

\end{document}
